\chapter{mainpage }
\hypertarget{md__projekt_2src_2main_2java_2budzet_2_doxyfile__z__osobna__strona___u_m_l_2docs_2mainpage}{}\label{md__projekt_2src_2main_2java_2budzet_2_doxyfile__z__osobna__strona___u_m_l_2docs_2mainpage}\index{mainpage@{mainpage}}
📊 Projekt ZTP – System Zarządzania Budżetem Domowym

🎯 Cel projektu Celem projektu jest stworzenie konsolowej aplikacji do zarządzania budżetem domowym, która umożliwia:\+


\begin{DoxyItemize}
\item ewidencjonowanie przychodów i wydatków,
\item analizę danych finansowych,
\item prognozowanie przyszłych kosztów,
\item generowanie raportów,
\item eksport danych do plików CSV,
\item automatyczne reagowanie na zmiany stanu budżetu.
\end{DoxyItemize}

🧩 Zastosowane wzorce projektowe W projekcie wykorzystano następujące wzorce:\+

🔹 Obserwator ~\newline
 Cel:\+ automatyczne powiadamianie o zmianach stanu budżetu ~\newline
 Przykład:\+ informowanie o przekroczeniu limitu wydatków ~\newline
 Pakiet:\+ obserwatorzy

🔹 Strategia ~\newline
 Cel:\+ możliwość wyboru algorytmu prognozowania w czasie działania programu ~\newline
 Pakiet:\+ prognozy

🔹 Fabryka ~\newline
 Cel:\+ centralizacja procesu tworzenia raportów ~\newline
 Pakiet:\+ raporty

🔹 Adapter ~\newline
 Cel:\+ integracja zewnętrznej biblioteki zapisu CSV z interfejsem systemu ~\newline
 Pakiet:\+ eksport

🗂 Struktura projektu 
\begin{DoxyCode}{0}
\DoxyCodeLine{projekt/}
\DoxyCodeLine{├──\ rdzen/\ \ \ \ \ \ \ \ \ \ \ \ \ \ \#\ logika\ domenowa\ (budżet,\ transakcje)}
\DoxyCodeLine{├──\ obserwatorzy/\ \ \ \ \ \ \ \#\ wzorzec\ Obserwator}
\DoxyCodeLine{├──\ prognozy/\ \ \ \ \ \ \ \ \ \ \ \#\ wzorzec\ Strategia}
\DoxyCodeLine{├──\ raporty/\ \ \ \ \ \ \ \ \ \ \ \ \#\ wzorzec\ Fabryka}
\DoxyCodeLine{├──\ eksport/\ \ \ \ \ \ \ \ \ \ \ \ \#\ wzorzec\ Adapter}
\DoxyCodeLine{├──\ docs/\ \ \ \ \ \ \ \ \ \ \ \ \ \ \ \#\ dokumentacja\ projektu}
\DoxyCodeLine{└──\ InterfejsKonsolowy.java}

\end{DoxyCode}


▶️ Uruchomienie projektu

Wymagania
\begin{DoxyItemize}
\item Java JDK 17 lub nowsza
\end{DoxyItemize}

Kompilacja 
\begin{DoxyCode}{0}
\DoxyCodeLine{javac\ *.java}

\end{DoxyCode}


Uruchomienie 
\begin{DoxyCode}{0}
\DoxyCodeLine{java\ InterfejsKonsolowy}

\end{DoxyCode}


Projekt nie wymaga zewnętrznych bibliotek.

👤 Instrukcja użytkownika (skrót) Aplikacja działa w trybie konsolowym i oferuje menu tekstowe umożliwiające:\+


\begin{DoxyItemize}
\item dodawanie transakcji,
\item przypisywanie kategorii,
\item generowanie raportów,
\item wykonywanie prognoz,
\item eksport danych do CSV.
\end{DoxyItemize}

Obsługa programu odbywa się poprzez wybór odpowiednich opcji menu.

👥 Podział pracy w zespole Karol Ziemak-\/ Rdzeń Systemu ~\newline
 Jakub Wierciszewski -\/ eksport + dokumentacja ~\newline
 Michał Szwabowicz-\/ Prognozy ~\newline
 Szymon Duchnowski-\/ Raporty + obserwatorzy + dokumentacja ~\newline


📄 Dokumentacja Szczegółowa dokumentacja projektu znajduje się w katalogu docs/\+ i obejmuje:\+


\begin{DoxyItemize}
\item opis wzorców projektowych,
\item diagramy UML,
\item instrukcję użytkownika i instalacji.
\end{DoxyItemize}

{\bfseries{Hierarchia klas generowana automatycznie przez Doxygena została pominięta, ponieważ diagram UML przedstawia pełniejszą strukturę relacji między klasami uczestniczącymi w zastosowanych wzorcach projektowych.}} 